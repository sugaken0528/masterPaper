\chapter{VGMLの実装}\label{cha:Implementation}

本研究では、\ref{cha:Function}節で示した、既存手法の課題を解決するために、VDM++仕様書における以下の4つの構文を生成する手法を提案する。
さらに、提案手法を既存手法に適用し、機械学習を活用したVDM++仕様書自動生成ツールVGMLを開発する。

\begin{itemize}
    \item クラス
    \item インスタンス変数定義
    \item 関数定義
    \item 操作定義
\end{itemize}

本研究で開発するVGMLの構造を、図\ref{}に示す。以降、4つの構文の生成手法について詳細を説明する。

\section{クラスへの対応}
本節では、自然言語仕様書からVDM++仕様書におけるクラスを生成する手法を提案する。具体的には、既存手法の処理部である変換部と機械学習部に対して以下の改良を行う。

\begin{itemize}
    \item 変換部の自然言語仕様書内の単語にパラメータを追加する処理の改良。
    \item 機械学習部の学習済みモデルを生成する処理を改良。
    \item 機械学習部の自然言語仕様書内の単語を分類する処理を改良。
\end{itemize}

以降、3つの改良について詳細を述べる。

\subsection{変換部の自然言語仕様書内の単語にパラメータを追加する処理の改良}

