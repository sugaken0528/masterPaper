\chapter{研究の準備}
\label{cha:Preparation}

本章では、VGMLを開発するために必要となる前提知識を説明する。

\section{VDM++}
\label{sec:vdm}

形式手法の1つに、VDM(Vienna Development Method)がある\cite{}。
VDMは、1970年代にIBMのウィーン研究所にてPL/Iコンパイラの正しさを検証するために開発された形式手法である。
1996年には、形式仕様記述言語VDM-SLが標準となっている\cite{}。
VDM++は、VDM-SLを基にオブジェクト指向拡張した言語であり、現在VDMの中では主流である。

本研究では、VDM++仕様書を自動生成する。
VDM++は、VDMTools\cite{}やVDMJ\cite{}などの支援ツールが揃っており、他の形式手法に比べ仕様の検証がしやすい利点がある。
表\ref{table:vdm_block}に、VDM++の定義ブロックと、定義ブロックに対応する日本語での呼び方を示す。

\begin{table}[t]
    \begin{center}      
        \caption{VDM++の定義ブロックと定義ブロックに対応する日本語での呼び方}\label{table:vdm_block}
        \begin{tabular}{c|c}
        VDM++の定義ブロック  & 日本語での呼び方 \\ \hline \hline
        class & クラス \\ \hline
        types	 & 型定義 \\ \hline
        values  & 定数定義 \\ \hline
        instance variables & インスタンス変数定義 \\ \hline
        operations & 操作定義 \\ \hline 
        functions  & 関数定義 \\ \hline 
        \end{tabular}
    \end{center}
\end{table}

本研究では、表\ref{table:vdm_block}におけるVDM++の定義ブロックであるクラス、型定義、定数定義、インスタンス変数定義、操作定義に対応する。

\begin{table}[t]
    \begin{center}      
        \caption{VDM++の各ブロックの構文}\label{table:vdm_syntax}
        \begin{tabular}{c|c|c}
        ブロック  & 構文 \\ \hline \hline
        class & 名前 \\ \hline
        types	 & public 名前 = real ;\\ \hline
        values  & 名前 = 値 ;\\ \hline
        instance variables & public 名前 : 型 ;\\ \hline
        operations & 名前 : () ==\textgreater real\\
                   & 名前()==\\
                   & return 0 ;\\ \hline 
        functions  & 名前 : () -\textgreater real \\
                   & 名前() == 0; \\ \hline 
        \end{tabular}
    \end{center}
\end{table}

\section{WordNet}
\label{sec:wordNet}
記述中

\section{Mecab}
\label{sec:mecab}
記述中

\section{TfidfVectorizer}
\label{sec:tfidf}
記述中

\section{ロジスティック回帰}
\label{sec:logistic}
記述中