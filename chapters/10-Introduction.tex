\chapter{はじめに}
\label{cha:Introduction}

社会におけるソフトウェアの重要性はますます高まっており、ソフトウェアのバグは社会に大きな影響を与えている\cite{}。
ソフトウェアにバグが混入する原因の1つに、ソフトウェア開発の上流工程において自然言語仕様書を使用していることが挙げられる。
自然言語には元来曖昧さが含まれており、自然言語で書かれたソフトウェア仕様書をプログラマが読むと、仕様書作成者の意図とは異なる
解釈をしてしまう場合がある。
そのため、プログラマが仕様書作成者の意図とは異なるプログラムの実装を行った結果、ソフトウェアにバグが混入する可能性がある。
この問題を解決するための方法として、ソフトウェア開発の上流工程において、形式手法を用いることが挙げられる。
形式手法を用いたソフトウェアの開発は、数理論理学をベースとした形式仕様記述言語によって記述されるため、
自然言語が持つ曖昧さを除いた、厳密な仕様書を作成することが可能である。

開発現場向けの形式手法の1つとして、VDM(Vienna Development method)\cite{}が存在する。
また、オブジェクト指向に基づいたモデル化を扱えるようにVDMの文法を変更した形式仕様記述言語VDM++が存在する。
VDM++のような形式仕様記述言語は、厳密な文法を持ち、かつ、データ型やシステムの不変条件などを書く必要があるため、記述が困難である。
さらに、従来この作業は、プログラマ個人の経験に依存しており、属人性が高いという問題がある。

そこで執行らは、自然言語仕様書内の単語に着目し、機械学習を用いてVDM++仕様書を生成するツールを開発した\cite{}。
この既存ツールは、自然言語仕様書を入力とし、自然言語仕様書から抽出した単語を、VDM++仕様書に必要である単語と、VDM++仕様書に必要でない単語に分類できる。
さらに、型定義と定数定義を記述したVDM++仕様書を生成できる。
しかし、自然言語仕様書から抽出した単語を、クラスやその他のブロック定義に分類することはできない。
そのため、既存ツールは、対応している構文が少なく、有用性が低い。

そこで本研究は、既存ツールの有用性の向上を目的として、自然言語仕様書から、クラス、インスタンス変数定義、操作定義に対応したVDM++仕様書を生成するための
手法を提案し、既存ツールに適用することでVDM++仕様書生成ツールVGMLを開発する。
まず、クラスへの対応として、本研究で新たに定義する概念レベルを自然言語仕様書内の単語に対して算出し、機械学習に必要なパラメータとして追加する。
さらに、機械学習を用いて自然言語仕様書内の単語を、VDM++仕様書に必要でない単語、VDM++仕様書に必要であるが、クラスの候補ではない単語、
VDM++仕様書に必要であり、かつ、クラスの候補である単語に分類することでクラスの候補である単語を抽出する。

次に、インスタンス変数定義への対応として、クラスの候補である単語と、クラスの候補である単語に接続する単語の概念レベルを比較することによって、
インスタンス変数定義の候補である単語を抽出する。
最後に、操作定義への対応として、VDM++仕様書に必要であるが、クラスの候補ではない単語の内、動詞を表す単語を抽出し、クラスの候補である単語と
抽出した動詞の自然言語仕様書内での関係を解析することによって操作定義の候補である単語を抽出する。

本論文の構成を、以下に示す。
第2章では、VGMLを開発するために必要となる前提知識を説明する。\\
第3章では、既存ツールについて説明する。\\
第4章では、VGMLについて説明する。\\
第5章では、VGMLに自然言語仕様書を適用した結果を示す。\\
第6章では、VGMLの有用性の評価と関連研究について述べる。\\
第7章では、本研究のまとめと今後の課題について述べる。

なお、本研究は、日本語で記述された文書を対象とする。
