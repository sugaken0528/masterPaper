\chapter{おわりに}\label{cha:Conclusion}

本研究では、既存ツールの有用性の向上を目的として、VDM++仕様書におけるクラス、インスタンス変数定義、操作定義の3つを生成する手法を提案した。
さらに、提案手法を既存ツールに適用し、機械学習を活用したVDM++仕様書自動生成ツールVGMLを開発した。
具体的には、まず、クラスへの対応として、本研究で新たに定義する概念レベルを自然言語仕様書内の単語に対して算出し、機械学習に必要なパラメータとして追加した。
さらに、機械学習を用いて自然言語仕様書内の単語を、UNNECESSARY、NONCLASS、CLASSの3つに分類することでクラスの候補である単語を抽出した。
次に、インスタンス変数定義への対応として、クラスの候補である単語と、クラスの候補である単語に接続する単語の概念レベルを比較することによって、
インスタンス変数定義の候補である単語を抽出した。
次に、操作定義への対応として、VDM++仕様書に必要であるが、クラスの候補ではない単語の内、動詞を表す単語を抽出し、クラスの候補である単語と
抽出した動詞の自然言語仕様書内での関係を解析することによって操作定義の候補である単語を抽出した。
最後に、抽出した各定義ブロックの候補である単語を、VDM++仕様書に記述することで、クラス、型定義、定数定義、インスタンス変数定義、操作定義を記述した
VDM++仕様書を生成した。

適用例として、本研究で開発したVGMLに対して仕様書Aと仕様書Bの2つの自然言語仕様書を適用した。
その結果、VGMLによって、自然言語仕様書から機械学習を用いてVDM++仕様書のクラス、型定義、定数定義、インスタンス変数定義、操作定義を自動生成できること、
かつ、生成したVDM++仕様書の構文が正しいことを確認できた。

また、有用性の評価のために、VGMLが生成したVDM++仕様書と人手によって記述したVDM++仕様書を比較し、
クラス、インスタンス変数定義、操作定義に対してF値を用いて評価した。
評価の結果、まず、VGMLは、自然言語仕様書から抽出した単語をクラスとしてVDM++仕様書に記述する際に、仕様書AでF値1.0、仕様書BでF値0.72
の精度で記述できることを確認した。
次に、VGMLは、自然言語仕様書から抽出した単語をインスタンス変数定義としてVDM++仕様書に記述する際に、仕様書AでF値0.62、仕様書BでF値0.61
の精度で記述できることを確認した。
最後に、VGMLは、自然言語仕様書から抽出した単語を操作定義としてVDM++仕様書に記述する際に、仕様書AでF値0.68、仕様書BでF値0.54
の精度で記述できることを確認した。
このことにより、VGMLは、教師データの自然言語仕様書と同じ自然言語仕様書と、異なる自然言語仕様書に対しても、
一定の割合でクラス、型定義、定数定義、インスタンス変数定義、操作定義を記述したVDM++仕様書を自動生成でき、
既存ツールから、対応するVDM++の構文を増やすことができたといえる。

以上から、VGMLは、既存ツールの有用性を向上できたといえる。

以下に、今後の課題を示す。

\begin{itemize}
	\item 関数定義に対応していない\\VGMLは、VDM++を構成する定義ブロックの1つである関数定義には対応していない。これによって、自然言語仕様書が対象とするシステムにおいて、振る舞いを表す名詞が、インスタンス変数を参照するか否かの判定ができないため、対応する必要がある。
	\item VDM++における制約条件に対応していない\\本研究は、自然言語仕様書からVDM++仕様書を生成することを目的としてる。しかし、VGMLが生成するVDM++仕様書は、VDM++仕様書において重要である事前条件、事後条件、不変条件といった制約条件の記述に対応できていない。これらの制約条件は、VDM++仕様書において重要な要素であるため、対応する必要がある。
	\item 型・定数定義の候補である単語をクラスの候補となる単語へ分類できていない\\VGMLは、抽出した型・定数定義の候補である単語を、クラスの候補となる単語へ分類できない。これによって、どのクラスが責任を持つ型定義および定数定義かの判定ができないため、対応する必要がある。
	\item VDM++仕様書のreal型以外の型に対応していない\\VGMLは、VDM++における型定義を生成する際に、実数型であるreal型しか出力できない。このため、型についてより厳密な仕様書を生成できないため、対応する必要がある。
	\item 操作定義の振る舞いの詳細について記述できていない\\VGMLは、自然言語仕様書が対象とするシステムにおいて、振る舞いを表す単語を操作定義としてVDM++仕様書に記述できる。しかし、その振る舞いの詳細を記述することはできない。これによって、生成した操作定義がどのような振る舞いを表すかをユーザが認識できないため、対応する必要がある。
\end{itemize}

